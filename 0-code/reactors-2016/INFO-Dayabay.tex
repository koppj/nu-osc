%pppp!TEX TS-program = pdflatex
\documentclass[12pt,oneside, a4paper]{article}
\pdfoutput=1

\ifx\pdfoutput\undefined
\usepackage[dvips,bookmarks=false]{hyperref}	% This is for arXiv.org
\else
\usepackage{hyperref}	% This is for pdftex
\fi
\hypersetup{colorlinks,bookmarksopen,bookmarksnumbered,citecolor=blue,
linkcolor=black,pdfstartview=FitH,urlcolor=blue}
\def\myurl#1#2{\href{http://#1}{#2}}
\def\hhref#1{\href{http://arxiv.org/abs/#1}{#1}} % in bibliography
\def\mhref#1{\href{mailto:#1}{#1}}		% email on title page

\oddsidemargin 0cm
%\evensidemargin 0cm
\textwidth 16cm
\textheight 23cm
\topmargin -0.8cm
\renewcommand{\baselinestretch}{1.1}


\usepackage{graphicx}
\usepackage{amssymb}
\usepackage{cite}
\usepackage{bm}
\usepackage{indentfirst}
\usepackage{amsmath}
%\usepackage{showkeys}
%\usepackage{eufrak}
%%%
%%%

\newcommand{\capdef}{}
\newcommand{\mycaption}[2][\capdef]{\renewcommand{\capdef}{#2}%
       \caption[#1]{{\footnotesize #2}}}


\newcommand{\be}{\begin{equation}}
\newcommand{\ee}{\end{equation}}

\newcommand{\draftnote}[1]{{\color{red}\bf #1}}

\begin{document}

%\begin{titlepage}

\hfill \today

\begin{center}

\vspace{1cm}
{\Large\bf Description of the Daya Bay fit used for NuFit 2.0}
\vspace{0.5cm}


%% \renewcommand{\thefootnote}{\fnsymbol{footnote}}
%% Thomas Schwetz$^{1,}$\footnote[1]{schwetz@mpi-hd.mpg.de}
%% \vspace{5mm}

based on data given in the Daya Bay talk at Neutrino 2014 

%{\it%
%$^{1}${Max-Planck-Institut f\"ur Kernphysik, Saupfercheckweg 1, 69117 Heidelberg, Germany}\\
%}

\vspace{5mm}
%\abstract{}

\end{center}
%\end{titlepage}


\renewcommand{\thefootnote}{\arabic{footnote}}
\setcounter{footnote}{0}

%\setcounter{page}{2}

The number of events (including oscillations) 
in the energy bin $i$ in the detector $d$ can be written
%
\begin{equation}
  N^d_i = \sum_r \frac{C^d}{L^2_{rd}} \int dE R_i(E) \phi(E) \sigma(E) P_{rd}(E) \,,
\end{equation}
% 
where the sum over $r$ is over all reactors, $L_{rd}$ is the
distance of reactor $r$ to detector $d$, and $R_i(E)$ is the response
function for the bin $i$. We assume a Gaussian energy resolution,
hence $R_i(E)$ can be written in terms of the error function, and we
assume that it is the same for all detectors, and identical binning is
used for all detectors. $\phi(E)$ is the reactor
flux (assumed to be the same for all reactors), $\sigma(E)$ is the
detection cross section, $P$ is the oscillation probability.
The factor $C^d$ contains the efficiencies and the DAQ time for each
detector, and includes the contributions from the ``DayaBay 6'' and
``DayaBay 8'' periods. Those data are given in the Neutrino2014 talk.

The 8 detectors are (somewhat artificially) divided into near and far
detectors. Let us write the total number of events in a given energy bin for
all the near detectors as
%
\begin{equation}\label{eq:ND}
  N^{ND}_i \equiv \sum_{d = ND} N^d_i = w_{ND}\, M_i \, \langle P \rangle_{i,ND} \,,
\end{equation}
where
\begin{align}
  & w_{ND} = \sum_{d=ND} \sum_r \frac{C^d}{L^2_{rd}} \,,\\
  & M_i = \int dE R_i(E) \phi(E) \sigma(E) \,,
\end{align}
and
\begin{equation}\label{eq:P}
\langle P \rangle_{i,ND} \equiv  \frac{1}{w_{ND} M_i}
\sum_{d=ND} \sum_r \frac{C^d}{L^2_{rd}} \int dE R_i(E) \phi(E) \sigma(E) P_{rd}(E) \,.
\end{equation}
%
The important obeservation is that within our assumptions $M_i$ is independent of $r$ and $d$.

In an obvious way, we can also write the events in a given bin $i$ in all far detectors:
%
\begin{equation}
  N^{FD}_i = w_{FD}\, M_i \, \langle P \rangle_{i,FD} \,,
\end{equation}
% 
with the same definitions, but the sum over $d$ runs now over all
far detectors. Note that $M_i$ is the same as in eq.~\eqref{eq:ND}.

\bigskip {\bf Assumption:} It is not clear how exactly the Daya Bay collaboration uses the ND data to predict the FD data in each bin. My assumption is that they use the ND data to determine $M_i$:
\begin{equation}
  M_i = \frac{O_i^{ND}}{w_{ND} \langle P \rangle_{i,ND}} \,,
\end{equation}
%
where $O_i^{ND}$ is the total number of observed  events in bin $i$ summed over all near 
detectors, and then use this to predict the FD data:
%
\begin{equation}\label{eq:FD}
  N^{FD}_i = \frac{w_{FD}}{w_{ND}}\, \frac{\langle P \rangle_{i,FD}}{\langle P \rangle_{i,ND}} 
  O_i^{ND} \,.  
\end{equation}

\bigskip
This equation is the basis for my fit. Now I use the following imput:
\begin{enumerate}
\item In the talk the FD spectrum expected in case of no oscillations
  is given, predicted from the ND data. This information is used to obtain 
  $O_i^{ND}$ by setting the probabilities in eq.~\eqref{eq:FD} to one:
\begin{equation}
  O_i^{ND}  = \frac{w_{ND}}{w_{FD}}
  \left(N^{FD}_i\right)_\text{no-osc}\,.
\end{equation}
  
\item Having determined $O_i^{ND}$ we can now use eq.~\eqref{eq:FD} in
  case of oscillations to predict $N^{FD}_i$ and compare it to the
  data given in the Neutrino2014 talk. To calculate the average
  probabilities in eq.~\eqref{eq:P} we take standard assumptions on
  $\phi(E)$ (including isotope composition of the cores) and the cross section.

\item Since the data is already background subtracted, the background
  is taken into account only in the uncertainty in each bin. 
  We include all the background components in the statistical
  error per bin. Furthermore the (correlated) uncertainty
  on the accidental background spectrum in included as a pull parameter.

\item Then there are two more pulls: 
  \begin{itemize}
  \item 
  Relative detector normalization: each detector 0.2\%.
  For 4 NDs and 4 FDs: $0.2\% / \sqrt{4}$.   
  Summing the contribution of ND + FD gives a factor $\sqrt{2}$. Hence we obtain an error of
  $\sqrt{2} \times 0.002 / 2$. 
\item
We include also an energy scale uncertainty of 0.35\%.
  \end{itemize}
\end{enumerate}

{\bf Remark:} In order to reproduce exactly the best fit point for
$\theta_{13}$ we re-scale $O^{ND}_i$ by a fudge factor of 0.999.

\end{document}
